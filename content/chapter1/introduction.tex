\section{Introduction} \label{sec:introduction} 

In the dynamic landscape of software architecture, the software development paradigms of
\gls{ca} and \gls{ns} have emerged as pivotal in addressing the multifaceted challenges of
software design, particularly in managing stability to achieve evolvability in software.
This paper delves into the synergy between these two architecture 'paradigms', each
contributing significantly to the contemporary discourse on software architectural
complexity.

Tracing the historical underpinnings of these concepts reveals the works of pioneers like
\textcite{d_mcilroy_nato_1968}, who was one of the first to discuss modular programming,
and \textcite{lehman_programs_1980}, who pointed out the importance of software evolution.
Contributions from \textcite{dijkstra_letters_1968} on structured programming and
\textcite{parnas_criteria_1972} on software modularity further cemented the foundation for
\gls{ca} and \gls{ns}. These historical insights contextualize the evolution of software
engineering principles and underscore the relevance of fostering maintainable and
evolvable software systems.

This paper outlines the insights from a design science research conducted by
\citename{koks_convergence_2023}{author}, exploring the significant benefits and practical
implications of integrating the strengths of \gls{ca} and \gls{ns} within the field of
software development \cite{koks_convergence_2023}. Besides the theoretical study of
comparing the principles and building blocks of both paradigms, the research included an
architectural design artiact, and a software artifact where the principles were applied and tested
in practice.  

The introduction is intended to set the stage and articulate the goal of this paper.
Section II lays out the theoretical background, focusing on the specific principles and
elements of each Software Design Paradigm while highlighting their unified concepts.
Section III delves into a detailed comparison of the principles and elements of CA and NS,
examining their similarities, differences, and their impact on the evolvability of
software constructs. Section IV explores the convergence of design elements between CA and
NS, providing a practical perspective on their integration. Section V discusses the
development and analysis of research artifacts, including the Expander Framework and Clean
Architecture Expander, to evaluate the convergence of the two theories in a practical
context. Section VI presents the research artifacts, detailing their construction and the
methodologies used to assess their effectiveness. Finally, Section VII concludes the paper
with a summary of findings, discussing the implications of the research and offering
recommendations for future work in the field of software architecture.