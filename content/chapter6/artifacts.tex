\section{The Research Artifacts}

\begin{figure*}[ht!]
  \centering
  \centerline{\includegraphics[scale=0.6]{figures/artifactOverview.pdf}}
  \caption[Schematic overview of the artifacts]{Schematic overview of the artifacts}
  \label{fig_overview_design}
\end{figure*}

The first artifact consists of two main components: the Clean Architecture Expander and
the Expander framework. The name of the Expander Framework, Pantha Rhei, was inspired by
the Greek philosopher \emph{Heraclitus}, who famously stated that \enquote{life is flux.}
The name reflects the artifact's perceived ability to cope with constant change in a
stable and evolvable manner. Users can interact with the Expander Framework using the
\gls{cli} command \enquote*{flux} in combination with several parameters.

As illustrated in Figure \ref{fig_overview_design}, the main task of the first artifact or
\enquote*{expand} the second artifact. By entering the correct command, the Expander
Framework loads the model being instantiated during the expansion process. Then, the
required expanders are prepared based on information available through the model. In the
case of this study, the Clean Architecture Expander. The Clean Architecture Expander
consists of a set of tasks and templates. When the Expander Framework executes the Clean
Architecture Expander, the model is instantiated into the generated artifact with the aid
of the templates.

The model is an instance of the meta-model. Consequently, the model can represent any
application as long as the meta-model is respected. In the case of this study, the model
represents the entities, attributes, relationships, and other characteristics of the
meta-model.

As a result, the second artifact (artifact II) allows a user to modify or maintain the
model used by the Expander Framework by exposing a Restful interface. This method
approaches the meta-circularity process, where an expansion process is used to update the
meta-model. Although not fully compliant with the theory of \gls{ns}, the Expander
Framework consists of the required tasks to update its own meta-model. This is illustrated
in Figure \ref{fig_overview_design} by the \enquote*{updates} arrow.



