\subsection{Naming Conventions} \label{naming_convention}

The following section introduces the naming conventions applied throughout the project.
While these conventions do not directly contribute to the stability aspects of the
software architecture, they serve an important role. By adhering to consistent and
descriptive naming patterns, it becomes easier to follow the structure of the code and
identify key components of the artifacts. These naming conventions help readers recognize
and map various elements to their corresponding roles within the \gls{ca} framework,
enhancing clarity and improving code comprehension without affecting the system's inherent
stability.

\textbf{[PROD]} is defined as \textit{The name of the product of the software.} \newline 
\textbf{[COMP]} is defined as \textit{The name of the Company that is considered the owner of the software. If
there is no company involved, this can be left blank.} \newline 
\textbf{[TECH]} is defined as \textit{The primary technology that is used by the component layer.} 

\begin{table}[H]
  \renewcommand{\arraystretch}{1.5}
    \footnotesize
    \caption{Naming convention component layers}
    \begin{tabular}{ l l }
    \hline
    \textbf{Layer} & \textbf{Convention} \\ 
    \hline
    Domain & \textbf{Project}: [PROD].Domain \\ & \textbf{Package}: [COMP].[PROD].Domain \\
    Application & \textbf{Project}: [PROD].Application \\ & \textbf{Package}: [COMP].[PROD].Application \\
    Presentation & \textbf{Project}: [PROD].Presentation.[TECH] \\ & \textbf{Package}: [COMP].[PROD].Presentation.[TECH] \\
    Infrastructure & \textbf{Project}: [PROD].Infrastructure.[TECH] \\ & \textbf{Package}: [COMP].[PROD].Infrastructure.[TECH]
    \\ \hline
    \end{tabular}

\label{table:component_naming_convention}
\end{table}

\textbf{[Verb]} is defined as \textit{The primary action that that class or interface is assosiated with.} \newline 
\textbf{[Noun]} is defined as \textit{The primary subject or object that that class or interface is assosiated with.} 

\begin{table}[ht]
  \renewcommand{\arraystretch}{1.5}
  \footnotesize
  \caption{Naming convention of recurring elements}
  \begin{tabular}{ p{0.17\linewidth} p{0.19\linewidth} p{0.11\linewidth} p{0.32\linewidth} }
  \hline
  \textbf{Layer} & \textbf{Element} & \textbf{Type} & \textbf{Convention} \\ \hline
  Presentation & Controller & class & [\textit{Noun}]Controller \\
  & ViewModel\-Mapper & class & [\textit{Noun}]ViewModel\-Mapper \\
  & Presenter & class & [\textit{Verb}][\textit{Noun}]Presenter \\
  & ViewModel & class & [\textit{Noun}]ViewModel \\

  Application & Boundary & class & [\textit{VerbNoun}]Boundary \\
  & Boundary  & interface & IBoundary \\
  & Gateway  & interface & I[\textit{Verb}]Gateway \\
  & Interactor  & interface & I[\textit{Verb}]Interactor \\
  & Interactor & class & [\textit{Verb}][\textit{Noun}]Interactor \\
  & Mapper  & interface & IMapper \\
  & Request\-ModelMapper & class & [\textit{Verb}][\textit{Noun}]Request\-ModelMapper \\
  & Presenter  & interface & IPresenter \\
  & Validator  & interface & IValidator \\
  & Validator & class & [\textit{Verb}][\textit{Noun}]Validator \\
  
  Infrastructure & Gateway & class & [\textit{Noun}]Repository \\

  Domain & Data Entity & class & [\textit{Noun}] \\ \hline

  \end{tabular}
  
  \label{table_element_naming_convention}
\end{table}