\section{Expansion with Clean Architecture}

The primary objective was to determine the degree of convergence between \gls{ca} and NS
Theory. The research consited out of several key objectives.

Besides the a comprehensive literature analysis, an architectural design was created,
which was fully and solely based on \gls{ca} principles. The findings from the literature
review were incorporated into this design, which served as the basis for the subsequent
development of the research artifacts.

In the artifact development phase, two artifacts were constructed to facilitate the study
of the convergence between \gls{ca} and NS Theories. The first artifact was the Expander
Framework and Clean Architecture Expander. These components were designed and implemented
based on the \gls{ca} design principles. The Clean Architecture Expander enabled the
parameterized instantiation of software systems that adhere to the principles and design
of \gls{ca}, while the Expander Framework served as a supporting system. It was responsible for
loading and orchestrating dependencies, managing models, and executing the Expander.

The second artifact was the Expanded Clean Architecture artifact. This artifact allowed
for the analysis of a RESTful API implementation and its alignment with \gls{ca}
principles and design, thereby providing a platform to evaluate the convergence of the two
theories in a practical context.

Finally, the analysis of combinatorics examined the artifacts for actual or potential
combinatorial effects. This analysis aimed to determine whether \gls{ca} and NS exhibit
convergence. The fundamental principles and architectural design of \gls{ca} were
considered throughout the analysis to ensure a comprehensive evaluation of the convergence
potential.

By pursuing these objectives, the research provides valuable insights into the interaction
between \gls{ca} and NS, particularly in terms of their potential convergence within the field
of software architecture.

This chapter outlines the construction of two artifacts. Both of these artifacts are
meticulously designed and developed in accordance with the design philosophy and
principles of \gls{ca} with strict adherence to the following requirements.