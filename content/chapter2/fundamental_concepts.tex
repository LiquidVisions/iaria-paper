\subsection{Unified concepts}
In this Section, we will examine concepts related to both \gls{ca} and \gls{ns}.
Understanding these concepts is crucial for executing the research and interpreting its
results.


\subsubsection{Modularity}
The original material of \textcite[82]{r_c_martin_clean_2018} describes a module as a 
piece of code encapsulated in a source file with a cohesive set of functions and data
structures. According to \textcite[22]{mannaert_normalized_2016}, modularity is a 
hierarchical or recursive concept that should exhibit high cohesion. While both design
approaches agree on the cohesiveness of a module's internal parts, there is a slight 
difference in granularity in their definitions.

\subsubsection{Cohesion}
\textcite[22]{mannaert_normalized_2016} consider cohesion as modules that exist out of
connected or interrelated parts of a hierarchical structure. On the other hand,
\textcite[118]{r_c_martin_clean_2018} discusses cohesion in the context of
components. He attributes the three component cohesion principles as crucial to grouping
classes or functions into cohesive components. Cohesion is a complex and dynamic process,
as the level of cohesiveness might evolve as requirements change over time. 

\subsubsection{Coupling}
Coupling is an essential concept in software engineering that is related to the degree of
interdependence among various software constructs. High coupling between components
indicates the strength of their relationship, creating an interdependent relationship
between them. Conversely, low coupling signifies a weaker relationship, where
modifications in one part are less likely to impact others. Although not always possible,
the level of coupling between the various modules of the system should be kept to a bare
minimum. Both \textcite[23]{mannaert_normalized_2016} and
\textcite[130]{r_c_martin_clean_2018} agree to achieve as much decoupling as possible.