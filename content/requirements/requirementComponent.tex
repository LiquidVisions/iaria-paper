\subsubsection{Component Architecture Requirements} \label{component_requirements}

The following requirements apply to the component architecture of both the Generator
artifact and the Generated artifact.

The component architecture is organized into separate Visual Studio projects for the
Domain, Application, Infrastructure, and Presentation layers. A detailed description of
these layers can be found in Section (((FIX FULLREF))). Each of these projects adheres to
the naming conventions described in Appendix (((FIX FULLREF))). Importantly, the
dependencies between component layers must follow an inward direction, aligning with
higher-level components as schematically illustrated in Figure
\ref{fig_modulair_components}. The dependencies cannot skip layers, ensuring a clear
hierarchical structure.

In terms of technology, the Domain and Application layers are designed to be independent
of any infrastructure technologies, such as web or database technologies. In contrast, the
Presentation Layer relies on various infrastructure technologies to facilitate interaction
with end-users. These technologies include Command Line Interfaces (CLIs), RESTful APIs,
and web-based solutions. Each dependency within the Presentation Layer is isolated and
managed in separate Visual Studio projects to ensure the system's stability and
evolvability.

The Infrastructure Layer may rely on additional components, such as databases or
filesystems, but similar to the Presentation Layer, each infrastructure dependency is
isolated and managed in its own Visual Studio project to maintain system stability and
evolvability. All layers within the component architecture utilize the C\# programming
language, explicitly targeting the .NET 7.0 framework.

Furthermore, the reuse of existing functionality or technology, such as packages, is
permitted only when it complies with the Liskov Substitution Principle (LSP) and makes use
of the NuGet open-source package manager. This ensures that any reused components align
with the overall design principles and maintain the flexibility and integrity of the
system.

By adhering to these requirements, the component architecture remains well-structured,
maintainable, and capable of evolving over time.
