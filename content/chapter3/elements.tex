\section{Comparing the elements elements}

In this section we compare the design elements of \gls{ca} and \gls{ns}, exploring their
convergence and application in software design. The discussion is anchored in the results
of the research  \citetitle{koks_convergence_2023} \cite{koks_convergence_2023}, which
examines the elements \gls{ca} and \gls{ns} mentioned in previous chapters, from both a
theoretical and practical perspective.

The Data Element from \gls{ns} and the Entity Element from \gls{ca} represent data objects
of the ontology or data schema, typically including attributes and relationship
information. While both can contain a complete set of attributes and relationships, the
Data Element of \gls{ns} may also be tailored to serve a specific set of information
required for a single task or use case. In \gls{ca}, these types of Data Elements are
explicitly specified as ViewModels, RequestModels, or Response Models.

The Interactor element of \gls{ca} and the Task and WorkFlow elements of \gls{ns} are all
responsible for encapsulating business rules. \gls{ns} has a more strict approach to
encapsulating the execution of business rules in Task Elements, as it is only allowed to
have a single execution of a business rule. Additionally, the WorkFlow element is
responsible for executing multiple tasks statefully and is highly convergable with the
Interactor element of \gls{ca}.

The convergence of the Controller element from \gls{ca} with \gls{ns} is highlighted by
its partial interchangeability with the Connector and Trigger elements in \gls{ns}. The
Controller Element is primarily responsible for interaction using protocols and
technologies involving the user interface, while the Connector and Trigger elements are
also intended to interact with other types of external systems.
 
The Gateway element of \gls{ca} and the Connector element of \gls{ns} communicate between
components by providing Data Version Transparent interfaces to provide Action Version
Transparency between these components.

The Presenter is responsible for preparing the ViewModel on the controller's behalf and
can be considered a Task or Workflow Element in the theories of \gls{ns}.

The Boundary element of \gls{ca} strongly converges with the Connector element of
\gls{ns}, as both are involved in communication between components and help ensure loose
coupling between these components. However, the Boundary element's scope seems more
specific, as this element usually separates architectural boundaries within the
application or component.

In the following table, we summarize the analysis in a tabular overview using the same
denotation used in Section \ref{subsec:converging_principles}.

\begin{table}[htbp]
    \caption{The convergence between \gls{ca} and \gls{ns} elements.} 
    \renewcommand{\arraystretch}{1.5}
    \centering
    \begin{tabular}{r|lllll}
    
        \textbf{\acrlong{ca}   } \textbf{   \rotatebox[origin=l]{90}{\acrlong{ns}}} & 
        \rotatebox[origin=l]{90}{Data Elements} & \rotatebox[origin=l]{90}{Task Element} &
        \rotatebox[origin=l]{90}{Flow Element} & \rotatebox[origin=l]{90}{Connector
        Element} & \rotatebox[origin=l]{90}{Trigger Element} \\
    \midrule
    
    
    Entity Element & \fullConvergence & \noConvergence & \noConvergence & \noConvergence & \noConvergence \\
    Interactor Element & \noConvergence & \fullConvergence & \fullConvergence & \noConvergence & \noConvergence \\
    RequestModel Element & \fullConvergence & \noConvergence & \noConvergence & \noConvergence & \noConvergence \\ 
    ResponseModel Element & \fullConvergence & \noConvergence & \noConvergence & \noConvergence & \noConvergence \\
    ViewModel Element & \fullConvergence & \noConvergence & \noConvergence & \noConvergence & \noConvergence \\
    Controller Element & \noConvergence & \noConvergence & \noConvergence & \npartialConvergence & \npartialConvergence \\
    Gateway Element & \noConvergence & \noConvergence & \noConvergence & \fullConvergence & \noConvergence \\
    Presenter Element & \noConvergence & \npartialConvergence & \npartialConvergence & \noConvergence & \noConvergence \\
    Boundary Element & \noConvergence & \noConvergence & \noConvergence & \fullConvergence & \noConvergence \\
    \bottomrule
    \end{tabular}
    \label{tab_convergence_elements_summarized}
\end{table}